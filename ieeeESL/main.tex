\documentclass[journal]{IEEEtran}

\usepackage{url}
\usepackage{breakurl}
\def\UrlBreaks{\do\/\do-} %hack to force url breaks on "-" in references
% \usepackage[breaklinks, colorlinks]
\usepackage{hyperref}
\renewcommand{\sectionautorefname}{Section}
\renewcommand{\subsectionautorefname}{Section}
\renewcommand{\subsubsectionautorefname}{Section}
\hypersetup{
    colorlinks,
    linkcolor={[rgb]{0,0,0.4}},
    citecolor={[rgb]{0,0,0.4}},
    urlcolor={[rgb]{0,0,1}},
    breaklinks=true,
}

\usepackage[ruled,vlined,linesnumbered]{algorithm2e}
\usepackage{comment}
% \usepackage{pdfcomment}
\usepackage{color}
\usepackage{graphicx}
\usepackage{caption}
\usepackage{subcaption}
\usepackage{threeparttable}
\usepackage{multirow}
\usepackage{booktabs}
\usepackage{verbatim}
\usepackage{epstopdf}
\usepackage{rotating}
\usepackage{listings}
\usepackage{listing}
\usepackage{paralist}
\usepackage{arydshln}
\let\labelindent\relax
\usepackage{enumitem,amssymb}
\usepackage{algpseudocode}% http://ctan.org/pkg/algorithmicx
\usepackage{balance}
\usepackage{endnotes}
\usepackage{xspace}
\usepackage{amsfonts}
\usepackage{pifont}
\usepackage{wasysym}
\usepackage{tikz}
% \usepackage{cite}
\usepackage[noadjust]{cite}
\renewcommand{\citepunct}{,\penalty\citepunctpenalty\,}
\renewcommand{\citedash}{--}

% to prevent "\pdfendlink ended up in different nesting level than \pdfstartlink" error.
% \let\oldbibitem\bibitem
% \def\bibitem{\vfill\oldbibitem}

\newcommand{\Tstrut}{\rule{0pt}{2.6ex}}         % = `top' strut
\newcommand{\Bstrut}{\rule[-0.9ex]{0pt}{0pt}}   % = `bottom' strut
\newcommand{\tick}{\ding{51}}
\newcommand{\cross}{\ding{55}}

\graphicspath{{./figs/}}

% correct bad hyphenation here
\hyphenation{op-tical net-works semi-conduc-tor}


\begin{document}
\title{Bare Demo of IEEEtran.cls\\ for IEEE Journals}

%
% author names and IEEE memberships
% note positions of commas and nonbreaking spaces ( ~ ) LaTeX will not break
% a structure at a ~ so this keeps an author's name from being broken across
% two lines.
% use \thanks{} to gain access to the first footnote area
% a separate \thanks must be used for each paragraph as LaTeX2e's \thanks
% was not built to handle multiple paragraphs
%
\author{Michael~Shell,~\IEEEmembership{Member,~IEEE,}
        John~Doe,~\IEEEmembership{Fellow,~OSA,}
        and~Jane~Doe,~\IEEEmembership{Life~Fellow,~IEEE}% <-this % stops a space
\thanks{M. Shell was with the Department
of Electrical and Computer Engineering, Georgia Institute of Technology, Atlanta,
GA, 30332 USA e-mail: (see http://www.michaelshell.org/contact.html).}% <-this % stops a space
\thanks{J. Doe and J. Doe are with Anonymous University.}% <-this % stops a space
\thanks{Manuscript received April 19, 2005; revised August 26, 2015.}}

% note the % following the last \IEEEmembership and also \thanks - 
% these prevent an unwanted space from occurring between the last author name
% and the end of the author line. i.e., if you had this:
% 
% \author{....lastname \thanks{...} \thanks{...} }
%                     ^------------^------------^----Do not want these spaces!
%
% a space would be appended to the last name and could cause every name on that
% line to be shifted left slightly. This is one of those "LaTeX things". For
% instance, "\textbf{A} \textbf{B}" will typeset as "A B" not "AB". To get
% "AB" then you have to do: "\textbf{A}\textbf{B}"
% \thanks is no different in this regard, so shield the last } of each \thanks
% that ends a line with a % and do not let a space in before the next \thanks.
% Spaces after \IEEEmembership other than the last one are OK (and needed) as
% you are supposed to have spaces between the names. For what it is worth,
% this is a minor point as most people would not even notice if the said evil
% space somehow managed to creep in.

% The paper headers
\markboth{IEEE EMBEDDED SYSTEMS LETTERS}%
{First \MakeLowercase{\textit{et al.}}: Title}
% The only time the second header will appear is for the odd numbered pages
% after the title page when using the twoside option.
% 
% *** Note that you probably will NOT want to include the author's ***
% *** name in the headers of peer review papers.                   ***
% You can use \ifCLASSOPTIONpeerreview for conditional compilation here if
% you desire.

% make the title area
\maketitle

% As a general rule, do not put math, special symbols or citations
% in the abstract or keywords.
\begin{abstract}
\input{abstract}
\end{abstract}

% Note that keywords are not normally used for peerreview papers.
\begin{IEEEkeywords}
IEEE, IEEEtran, journal, \LaTeX, paper, template.
\end{IEEEkeywords}

% For peer review papers, you can put extra information on the cover
% page as needed:
% \ifCLASSOPTIONpeerreview
% \begin{center} \bfseries EDICS Category: 3-BBND \end{center}
% \fi
%
% For peerreview papers, this IEEEtran command inserts a page break and
% creates the second title. It will be ignored for other modes.
\IEEEpeerreviewmaketitle


\section{Introduction}
\label{s:intro}

\begin{figure}[h]
	\centering
	\includegraphics[width=0.8\linewidth]{sample}
	\caption{Sample}
	\label{f:sample}
\end{figure}

see this~\cite{latex}
\section{Background}
\label{s:backg}
\input{design}
\input{eval}
\section{Conlcusion}
\label{s:concl}
\input{ack}


% \appendices
% \section{}
% Appendix one text goes here.
% \section{}
% Appendix two text goes here.

% Can use something like this to put references on a page
% by themselves when using endfloat and the captionsoff option.
\ifCLASSOPTIONcaptionsoff
  \newpage
\fi

\bibliographystyle{IEEEtran}
\bibliography{references}

\end{document}


